\cleardoublepage
\chapter*{\listofabbrevname}
\phantomsection
\addcontentsline{toc}{chapter}{\listofabbrevname}

\begin{acronym}[KolikMista]

	\acro{ReLU}
		[ReLU]
		{Usmernená lineárna jednotka (Rectified Linear Unit)}

	\acro{CNN}
		[CNN]
		{Konvolučné neurónové siete (Convolutional neural network)}

	\acro{R-CNN}
		[R-CNN]
		{Konvolučné neurónové siete založené na regiónoch (Region-based CNN)}

	\acro{RoI}
		[RoI]
		{Regióny záujmu (Regions of Interest)}

	\acro{SVM}
		[SVM]
		{Metódy podporných vektorov (Support Vector Machine)}

	\acro{RPN}
		[RPN]
		{Sieť návrhu regiónov (Region Proposal Network)}

	\acro{YOLO}
		[YOLO]
		{You only look once}
	
	\acro{IoU}
		[IoU]
		{Pomer prieniku voči zjednoteniu (Intersection over Union)}
	
	\acro{tkinter}
		[tkinter]
		{Rozhranie toolkitu Tcl/Tk (Tk interface)}

	\acro{csv}
		[csv]
		{Súbor s čiarkou rozdelenými hodnotami (Comma Separated Values)}

	\acro{GUI}
		[GUI]
		{Grafické užívateľské rozhranie (Graphical User Interface)}

	\acro{mAP}
		[mAP]
		{Stredná priemerná presnosť (Mean Average Precision)}

	\acro{KNN}
		[KNN]
		{K najbližších susedov (K Nearst Neighbours)}

	\acro{NMS}
		[NMS]
		{Potlačenie nemaximálnej hodnoty (Non-Maximum Suppression)}

	% \acro{zkTemp}		% název
	% 	[Šířka levého sloupce Seznamu symbolů a zkratek]								% zkratka
	% 	{je určena šířkou parametru prostředí \texttt{acronym} (viz řádek~1 výpisu zdrojáku na~str.\,\pageref{lst:zkratky})}
	% 										% rozvinutí zkratky

	% \acro{zkDummy}
	% 	[KolikMista]
	% 	{pouze ukázka vyhrazeného místa}

	% \acro{DSP}		% název/zkratka
	% 	{číslicové zpracování signálů -- Digital Signal Processing}
	% 										% rozvinutí zkratky
	% %%% bsymfvz
	% \acro{symfvz}						% název
	% 	[\ensuremath{f_\textind{vz}}] % symbol
	% 	{vzorkovací kmitočet}					% popis
	% %%% esymfvz

\end{acronym}
