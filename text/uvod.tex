\chapter*{Úvod}
\phantomsection
\addcontentsline{toc}{chapter}{Úvod}

V súčasnej dobe hrá počítačové videnie dôležitú rolu v mnohých aplikáciách. Táto práca sa zameriava na jeho využitie pri návrhu systému schopného detekovať, trasovať a klasifikovať objekty lietajúce na oblohe. Cieľom je vytvoriť robustný systém schopný nasadenia v reálnych nepriaznivých podmienkach so schopnosťou detekcie a klasifikácie rôznych typov živých aj umelo vytvorených lietajúcich objektov. Je požadovaná možnosť použitia rôznych prístupov spracovania vstupných dát a porovnanie ich výkonu a presnosti. Pri návrhu systému je nutné zvážiť obmedzený výkon výpočtovej jednotky systému a voliť jednoduchšie.

Systém bude získavať obrazové dáta pomocou statickej kamery, sledujúcej časť oblohy. Časť obrazu môže obsahovať aj horizont. Kamera v bude reálnom čase zachytávať sekvenciu snímkov. Na nich systém musí rozpoznať pohybuhúce sa objekty na pozadí ktoré sa môže s postupom času tiež pohybovať či meniť, či už kvôli pohybujúcim oblakom, alebo zmene svetelných podmienok.

Detekované objekty budú klasifikované do piatich tried: vták, hmyz, lietadlo, helikoptéra, dron. Kvôli rôznym farbám rôznych inštancii týchto objektov, môžu mať objekty vyšší ale aj nižší jas ako pozadie.

Táto práca bude postupovať od teoretických základov do konkrétneho návrhu a implementácie popísaného systému počítačového videnia.

Prvá kapitola sa venuje základným princípom detekcie objektov v obraze. Sú tu preskúmané existujúce metódy s porovnaním ich výhod a nevýhod, čo povedie k výberu vhodných metód pre navrhovaný systém.

V druhej kapitole bude popísané získavanie charakteristík popisujúcich jednotlivé detekované objekty. Zohľadňujú sa tu aspekty ako tvar, farba a ďalšie príznaky, na ktorých záleží klasifikácia v ďalšom kroku.

Na základe popisu objektov bude v ďalšej kapitole popísaná problematika klasifikácie, teda rozpoznávania a zaradenia detekovaných objektov do definovaných tried podľa ich typu (lietadlo, helikoptéra, vták, hmyz, \dots).

Štvrtá kapitola predstaví moderné metódy integrujúce detekciu a klasifikáciu do jedného procesu. Cieľom je optimalizácia efektivity a jednoduchosti architektúry systému.

Následne bude navrhnutý a odôvodnený výber hardvérových komponentov systému. Budú porovnané mikropočítače vhodné na aplikáciu v tejto práci a zvolené vhodné periférie na získavanie dát a užívateľské rozhranie.

Pre správne naučenie modelov použitých v systéme je nevyhnutné vytvoriť dostatočne rozsiahlu a správne navrhnutú anotovanú množinu dát, v prípade algoritmov počítačového videnia ide o databázu snímkov s označenými a triedenými objektami. Šiesta kapitola sa zameriava na tvorbu tejto databázy (datasetu) a jej validáciu.

V poslednej kapitole bude zdokumentovaný návrh softvéru pre systém detekcie, klasifikácie a trasovania objektov. V tejto kapitole bude priblížená štruktúra aplikácie, nástroje použité pri vývoji a návrh jej užívateľského rozhrania.

% Úvod studentské práce, např\,\dots
% 
% Nečíslovaná kapitola Úvod obsahuje \uv{seznámení} čtenáře s~problematikou práce.
% Typicky se zde uvádí:
% (a) do jaké tematické oblasti práce spadá, (b) co jsou hlavní cíle celé práce a (c) jakým způsobem jich bylo dosaženo.
% Úvod zpravidla nepřesahuje jednu stranu.
% Poslední odstavec Úvodu standardně představuje základní strukturu celého dokumentu.
% 
% Tato práce se věnuje oblasti \acs{DSP} (\acl{DSP}), zejména jevům, které nastanou při nedodržení Nyquistovy podmínky pro \ac{symfvz}.%
% \footnote{Tato věta je pouze ukázkou použití příkazů pro sazbu zkratek.}
% 
% Šablona je nastavena na \emph{dvoustranný tisk}.
% Nebuďte překvapeni, že ve vzniklém PDF jsou volné stránky.
% Je to proto, aby důležité stránky jako např.\ začátky kapitol začínaly po vytisknutí a svázání vždy na pravé straně.
% %
% Pokud máte nějaký závažný důvod sázet (a~zejména tisknout) jednostranně, nezapomeňte si přepnout volbu \texttt{twoside} na \texttt{oneside}!