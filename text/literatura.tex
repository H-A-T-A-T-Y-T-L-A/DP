% Pro sazbu seznamu literatury použijte jednu z následujících možností

%%%%%%%%%%%%%%%%%%%%%%%%%%%%%%%%%%%%%%%%%%%%%%%%%%%%%%%%%%%%%%%%%%%%%%%%%
%1) Seznam citací definovaný přímo pomocí prostředí literatura / thebibliography

\begin{thebibliography}{99}
	
	\bibitem{Jurecka2021} % predošlá diplomová práca
		JUREČKA, Tomáš. Detekce a klasifikace létajících objektů. Brno, 2021. Diplomová práca.

	\bibitem{Janousek2018} % Algoritmy detekcie lietajúcich objektov, príspevok PIERS
		JANOUŠEK, Jiří, Jan NOVOTNÝ, Petr MARCOŇ a Anna SIRUCKOVA. Algorithms for Flying Object Detection. In: 2018 Progress In Electromagnetics Research Symposium (PIERS-Toyama) [online]. Toyma, Japonsko, 2018, s. 782-783 [cit. 2024-01-03]. ISBN 978-4-8855-2316-8. ISSN 1559-9450. Dostupné z: doi:10.23919/PIERS.2018.8598196

	\bibitem{Morse1998/1} % Prahovanie, prednáška online
		MORSE, Bryan. Lecture 4: Thresholding [online]. Brigham, 1998 [cit. 2024-01-03]. Dostupné z: \url{https://homepages.inf.ed.ac.uk/rbf/CVonline/LOCAL_COPIES/MORSE/threshold.pdf}. Brigham Young University.

	\bibitem{Morse1998/2} % Popis objektu - radiometrické deskriptory, prednáška online
		MORSE, Bryan. Lecture 9: Shape Description (Regions) [online]. Brigham, 1998 [cit. 2024-01-03]. Dostupné z: \url{https://homepages.inf.ed.ac.uk/rbf/CVonline/LOCAL_COPIES/MORSE/region-props-and-moments.pdf}. Brigham Young University.

	\bibitem{Buhl2023} % Prahovanie, web online
		BUHL, Nikolaj. Image Thresholding in Image Processing. Encord [online]. 2023 [cit. 2024-01-03]. Dostupné z: \url{https://encord.com/blog/image-thresholding-image-processing/}

	\bibitem{Samina2023} % Cannyho detektor, web online
		SAMINA. EDUCATIVE, INC. What is Canny edge detection? EDUCATIVE, INC. Educative [online]. Educative, 2023 [cit. 2024-01-03]. Dostupné z: \url{https://www.educative.io/answers/what-is-canny-edge-detection}

	\bibitem{OpenCV2023} % Odčítanie pozadia, dokumentácia OpenCV
		OPEN SOURCE COMPUTER VISION. How to Use Background Subtraction Methods [online]. 2023 [cit. 2024-01-03]. Dostupné z: \url{https://docs.opencv.org/4.x/d1/dc5/tutorial_background_subtraction.html}

	\bibitem{Leung2021} % Kreslenie CNN, web online
		LEUNG, Kenneth. How to Easily Draw Neural Network Architecture Diagrams [online]. In: . [cit. 2024-01-03]. Dostupné z: \url{https://towardsdatascience.com/how-to-easily-draw-neural-network-architecture-diagrams-a6b6138ed875}

	\bibitem{Erbo2017} % Geometrické momenty a invarianty, článok online
		LI, Erbo, Yazhou HUANG, Dong XU a Hua LI. Shape DNA: Basic Generating Functions for Geometric Moment Invariants [online]. Cornell University, 2017, 3-8 [cit. 2024-01-03]. Dostupné z: doi:10.48550/arXiv.1703.02242
% \bibitem{sr72/2017}
% 	VYSOKÉ UČENÍ TECHNICKÉ V~BRNĚ.
% 	\emph{Směrnice č.\,72/2017, Úprava, odevzdávání a~zveřejňování závěrečných prací.}
% 	Online. Brno: VUT v~Brně, 2017.
% 	Úplné znění ke dni 11.\,4.\,2022.
% 	Dostupné z:\\
% 	{\small
% 	\url{https://www.vut.cz/uredni-deska/vnitrni-predpisy-a-dokumenty/smernice-c-72-2017-uprava-odevzdavani-a-zverejnovani-zaverecnych-praci-d161410}.}
% 	[cit.\ 2023-09-27].
%  
%  \bibitem{CSN_ISO_690-2022}
%      ÚŘAD PRO TECHNICKOU NORMALIZACI, METROLOGII A~STÁTNÍ ZKUŠEBNICTVÍ.
%      ČSN ISO 690:2022 (01 0197), \emph{Informace a dokumentace -- Pravidla pro bibliografické odkazy a~citace informačních zdrojů.}
%      Čtvrté vydání. Praha, 2022.
%  
%  \bibitem{CSN_ISO_7144-1997}
%      ÚŘAD PRO TECHNICKOU NORMALIZACI, METROLOGII A~STÁTNÍ ZKUŠEBNICTVÍ.
%      ČSN ISO 7144 (010161), \emph{Dokumentace -- Formální úprava disertací a~podobných dokumentů.}
%  %    24 stran.
%      Praha, 1997.
%  
%  \bibitem{CSN_ISO_31-11}
%      ÚŘAD PRO TECHNICKOU NORMALIZACI, METROLOGII A~STÁTNÍ ZKUŠEBNICTVÍ.
%      ČSN ISO 31-11, \emph{Veličiny a~jednotky -- část 11: Matematické znaky a~značky používané ve fyzikálních vědách a~v~technice.}
%      Praha, 1999.
%  
%  \bibitem{Farkasova23:CSNISO6902022komentar}
%  	FARKAŠOVÁ, B. et al.
%  	\emph{Výklad normy ČSN ISO 690:2022 (01 0197) účinné od 1.\,12.\,2022}.
%  	 Online. První vydání. 2023.
%  	Dostupné~z:
%  	\url{https://www.citace.com/Vyklad-CSN-ISO-690-2022.pdf}.
%  	[cit.\,2023-09-27].
%  
%  \bibitem{pravidla}
%      \emph{Pravidla českého pravopisu}.
%  	1.\ vydání. Olomouc: FIN, 1998.\\
%  	\mbox{ISBN 80-86002-40-3}.
%  
%  \bibitem{Walter1999}
%  	WALTER, G.\,G. a SHEN, X.
%  	\emph{Wavelets and Other Orthogonal Systems}.
%  	2.\,vydání, Boca Raton: Chapman\,\&\,Hall/CRC, 2000.
%  	ISBN 1-58488-227-1
%  
%  \bibitem{Svacina1999IEEE}
%  	SVAČINA, J.
%  	Dispersion Characteristics of Multilayered Slotlines -- a Simple Approach.
%  	\emph{IEEE Transactions on Microwave Theory and Techniques}.
%  	1999, vol.\,47, no.\,9, s.\,1826--1829. ISSN 0018-9480.
%  
%  \bibitem{RajmicSysel2002}
%      RAJMIC, P. a SYSEL, P.
%      Wavelet Spectrum Thresholding Rules.
%      In: \emph{Proceedings of the International Conference Research in Telecommunication Technology}.
%      Žilina: Žilina University, 2002. s.\,60--63. ISBN 80-7100-991-1.
%  
\end{thebibliography}


%%%%%%%%%%%%%%%%%%%%%%%%%%%%%%%%%%%%%%%%%%%%%%%%%%%%%%%%%%%%%%%%%%%%%%%%%
%%2) Seznam citací pomocí BibTeXu
%% Při použití je nutné v TeXnicCenter ve výstupním profilu aktivovat spouštění BibTeXu po překladu.
%% Definice stylu seznamu
%\bibliographystyle{unsrturl}
%% Pro českou sazbu lze použít styl czechiso.bst ze stránek
%% http://www.fit.vutbr.cz/~martinek/latex/czechiso.tar.gz
% \bibliographystyle{Czech}
%% Vložení souboru se seznamem citací
% \bibliography{text/literatura}
%
%% Následující příkaz je pouze pro ukázku sazby literatury při použití BibTeXu.
%% Způsobí citaci všech zdrojů v souboru literatura.bib, i když nejsou citovány v textu.
% \nocite{*}